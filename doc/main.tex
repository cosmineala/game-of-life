
\documentclass[11pt ,A4]{article}
% \usepackage{amsmath} % \usepackage is a command that allows you to add functionality to your LaTeX code
\usepackage[a4paper, total={7in, 9in}]{geometry}
% \usepackage[sfdefault]{roboto}

\usepackage[none]{hyphenat}
% \usepackage{url}
\usepackage[colorlinks=true, allcolors=blue]{hyperref}
\usepackage{indentfirst}
\setlength{\parindent}{0.5cm}

\title{Jocului Vietii\\Proiectul 6}
\author{Capisizu Cosmin Louis\\IS1.1} % Sets authors name
% \date{\today}

\begin{document}
    \maketitle % creates title using information in preamble (title, author, date)
    \pagebreak

    \section{Cerinta} % creates a section
        \paragraph{}
            Proiectarea si implementarea unui sistem multi-agent pentru simularea jocului vietii, game of life. Se va defini si analiza cel putin un joc de tip automat celular diferit de cel din programul demonstrativ pentru jocul vietii. Pentru jocul vietii se vor defini si experimenta si alte configuratii initiale, pe langa cele predefinite in programul demonstrativ.

    \section{Prezentarea jocului}

        \paragraph{}Jocul Vietii, cunoscut si sub numele de Viata, este un automat celular conceput de matematicianul britanic John Horton Conway in 1970. Este un joc fara jucatori, ceea ce inseamna ca evolutia sa este determinata de starea sa initiala, nefiind nevoie de alte informatii.

        \paragraph{} Se interactioneaza cu Jocul Vietii creand o configuratie initiala si observand cum evolueaza. Este Turing complet si poate simula un constructor universal sau orice alta masina Turing.

        \paragraph{} Universul Jocului Vietii este o retea ortogonala infinita, bidimensionala, de celule patrate, fiecare dintre acestea fiind intr-una dintre cele doua stari posibile, vie sau moarta (sau populata si, respectiv, nepopulata). Fiecare celula interactioneaza cu cei opt vecini ai sai, care sunt celulele care sunt adiacente orizontal, vertical sau diagonal.

        \paragraph{} La fiecare pas in timp, au loc urmatoarele tranzitii
        \begin{itemize}
            \item Orice celula vie cu mai putin de doi vecini vii moare, ca prin subpopulare.
            \item Orice celula vie cu doi sau trei vecini vii traieste in generatia urmatoare.
            \item Orice celula vie cu mai mult de trei vecini vii moare, ca prin suprapopulare.
            \item Orice celula moarta cu exact trei vecini vii devine o celula vie, ca prin reproducere.
        \end{itemize}

        \paragraph{} La fiecare pas in timp, au loc urmatoarele tranzitii Aceste reguli, care compara comportamentul automatului cu viata reala, pot fi condensate in urmatoarele
        \begin{itemize}
            \item Orice celula vie cu doi sau trei vecini vii supravietuieste.
            \item Orice celula moarta cu trei vecini vii devine o celula vie.
        \end{itemize}

        \paragraph{} Toate celelalte celule vii mor in generatia urmatoare. In mod similar, toate celelalte celule moarte raman moarte.

        \paragraph{} Modelul initial constituie samanta sistemului. Prima generatie este creata prin aplicarea regulilor de mai sus simultan la fiecare celula din samanta, vie sau moarta; nasterile si decesele au loc simultan, iar momentul discret in care se intampla acest lucru se numeste uneori capusa. Fiecare generatie este o functie pura a celei precedente. Regulile continua sa fie aplicate in mod repetat pentru a crea generatii viitoare.

    \section{Problema studiata}
        \paragraph{} In cadrul acestui proiect problema consta in implementarea si stimularea agentilor din cadrul jocului vietii.

    \section{Metoda de rezolvare folosita}

    \section{Functionarea programului pentru experimentare}

    \section{Cazuri experimentale}

    \section{Probleme intalnite}

        \paragraph{} Simularile multi-agent prezinta multe obstacole si probleme. Principalele dificultati in cadrul acestui proiect sunt:
        \begin{itemize}
            \item Implementarea sistemului ce retine si modifica starea agentilor. Acest sistem poate fi implementat folosind diferite structuri de date, cum ar fi: matricile sau listele. Principalul avantaj al matricii este accesul direct al al oricarui agent dar si al vecinilor acestuia, un mare desavantaj este fapul ca odata cu cresterea dimensiunii sumularii creste si marimea matricii. Simularea folosind liste permite o scalare mai usoara si o separare dintre numarul total de agenti posibil si marimea mediului de simulare. Cel mai mare dezavantaj consta in faptul ca anumite operatii sunt mult mai greu de implementat si necesita mult mai multi pasi de executie. In cazu acesui proiect matricea a fost solosita ca si mijloc de stocare al datelor.

            \item Implementarea si unei interfete grafice ce permite interactiunea directa cu simularea. Interfata grafica si simularea necesita arhitecturi total diferite. Deci simularea sistemului este punctul de referinta, interfata grafica afiseaza si modifica datele la cererea utilizatorului.
        \end{itemize}

    \section{Modificarile si imbunatatirile necesare}

    \section{Referinte}

        \begin{itemize}
            \item \href{https://en.wikipedia.org/wiki/Conway%27s_Game_of_Life}{wiki | Conway's Game of Life}
        \end{itemize}


\end{document} 